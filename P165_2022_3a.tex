\documentclass[12pt]{article}
%% arXiv paper template by Flip Tanedo
%% last updated: Dec 2016



%%%%%%%%%%%%%%%%%%%%%%%%%%%%%
%%%  THE USUAL PACKAGES  %%%%
%%%%%%%%%%%%%%%%%%%%%%%%%%%%%

\usepackage{amsmath}
\usepackage{amssymb}
\usepackage{amsfonts}
\usepackage{graphicx}
\usepackage{xcolor}
\usepackage{nopageno}
\usepackage{enumerate}
\usepackage{parskip}


\renewcommand{\thesection}{}
\renewcommand{\thesubsection}{\arabic{subsection}}

%%%%%%%%%%%%%%%%%%%%%%%%%%%%%%%%%%%%%%%%%%%%%%%
%%%  PAGE FORMATTING and (RE)NEW COMMANDS  %%%%
%%%%%%%%%%%%%%%%%%%%%%%%%%%%%%%%%%%%%%%%%%%%%%%

\usepackage[margin=2cm]{geometry}   % reasonable margins

\graphicspath{{figures/}}	        % set directory for figures

% for capitalized things
\newcommand{\acro}[1]{\textsc{\MakeLowercase{#1}}}    

\numberwithin{equation}{section}    % set equation numbering
\renewcommand{\tilde}{\widetilde}   % tilde over characters
\renewcommand{\vec}[1]{\mathbf{#1}} % vectors are boldface

\newcommand{\dbar}{d\mkern-6mu\mathchar'26}    % for d/2pi
\newcommand{\ket}[1]{\left|#1\right\rangle}    % <#1|
\newcommand{\bra}[1]{\left\langle#1\right|}    % |#1>
\newcommand{\Xmark}{\text{\sffamily X}}        % cross out

\let\olditemize\itemize
\renewcommand{\itemize}{
  \olditemize
  \setlength{\itemsep}{1pt}
  \setlength{\parskip}{0pt}
  \setlength{\parsep}{0pt}
}


% Commands for temporary comments
\newcommand{\comment}[2]{\textcolor{red}{[\textbf{#1} #2]}}
\newcommand{\flip}[1]{{\color{red} [\textbf{Flip}: {#1}]}}
\newcommand{\email}[1]{\texttt{\href{mailto:#1}{#1}}}

\newenvironment{institutions}[1][2em]{\begin{list}{}{\setlength\leftmargin{#1}\setlength\rightmargin{#1}}\item[]}{\end{list}}


\usepackage{fancyhdr}		% to put preprint number



% Commands for listings package
%\usepackage{listings}      % \begin{lstlisting}, for code
%
% \lstset{basicstyle=\ttfamily\footnotesize,breaklines=true}
%    sets style to small true-type



%%%%%%%%%%%%%%%%%%%
%%%  HYPERREF  %%%%
%%%%%%%%%%%%%%%%%%%

%% This package has to be at the end; can lead to conflicts
\usepackage{microtype}
\usepackage[
	colorlinks=true,
	citecolor=black,
	linkcolor=black,
	urlcolor=green!50!black,
	hypertexnames=false]{hyperref}





\begin{document}


\begin{center}

    {\Large \textsc{Short HW 3}:
    \textbf{Cross Sections}}
    
\end{center}

\vskip .4cm

\noindent
\begin{tabular*}{\textwidth}{rl}
	\textsc{Course:}& Physics 165, \emph{Introduction to Particle Physics} (2022)
	\\
	\textsc{Instructor:}& Prof. Flip Tanedo (\email{flip.tanedo@ucr.edu})
	\\
	\textsc{Due by:}& \textbf{Thursday}, April 14
\end{tabular*}



\section{Cross Section Limits}

Cross section measure the likelihood of a specific scattering process. In this problem, we try to build some intuition for the meaning of a cross section by drawing on classical physics. For this problem, the scattering is assumed to mean \emph{any kind of deflection of the probe relative to its motion if the target were not there}.

\subsection{Geometric Cross Section}

Suppose you are shooting tiny pellets at a billiard ball. What is the approximate classical cross section of the billiard ball?  For this problem you can use human-scale units like centimeters. Write a formula and give an order of magnitude answer with units.

\subsection{Charged Billiard Ball with Charged Probe}

Now suppose the tiny pellets had a small positive charge and the billiard ball had a small negative charge. Does the cross section go up, down, or stay the same?

\subsection{Charged Billiard Ball with Charged Probe, Too}

Same as the previous question, but now both the pellets and the billiard ball have positive charge. Does the cross section go up, down, or stay the same relative to the uncharged case? Does the cross section go up, down, or stay the same relative to the opposite-charge case?

\subsection{Cross section for a long-range force}

No calculation necessary\footnote{You do not have to do a calculation, but you may want to refer back to the discussion of scattering in classical mechanics to remind yourself of what's going on here.}. What is the cross section for an electron to scatter off a proton? The answer may cause you to pause, if it looks unusual, explain what it means. 


\textsc{Comment:} This question is a little tricky, if you are confused then articulate why you are confused and we can talk about it on Thursday! Do \emph{not} spend more than 30 minutes on this problem, you are \emph{not} meant to (re-)derive the classical central potential cross section.

\end{document}