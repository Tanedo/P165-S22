\documentclass[12pt]{article}
%% arXiv paper template by Flip Tanedo
%% last updated: Dec 2016



%%%%%%%%%%%%%%%%%%%%%%%%%%%%%
%%%  THE USUAL PACKAGES  %%%%
%%%%%%%%%%%%%%%%%%%%%%%%%%%%%

\usepackage{amsmath}
\usepackage{amssymb}
\usepackage{amsfonts}
\usepackage{graphicx}
\usepackage{xcolor}
\usepackage{nopageno}
\usepackage{enumerate}
\usepackage{parskip}


\renewcommand{\thesection}{}
\renewcommand{\thesubsection}{\arabic{subsection}}

%%%%%%%%%%%%%%%%%%%%%%%%%%%%%%%%%%%%%%%%%%%%%%%
%%%  PAGE FORMATTING and (RE)NEW COMMANDS  %%%%
%%%%%%%%%%%%%%%%%%%%%%%%%%%%%%%%%%%%%%%%%%%%%%%

\usepackage[margin=2cm]{geometry}   % reasonable margins

\graphicspath{{figures/}}	        % set directory for figures

% for capitalized things
\newcommand{\acro}[1]{\textsc{\MakeLowercase{#1}}}    

\numberwithin{equation}{section}    % set equation numbering
\renewcommand{\tilde}{\widetilde}   % tilde over characters
\renewcommand{\vec}[1]{\mathbf{#1}} % vectors are boldface

\newcommand{\dbar}{d\mkern-6mu\mathchar'26}    % for d/2pi
\newcommand{\ket}[1]{\left|#1\right\rangle}    % <#1|
\newcommand{\bra}[1]{\left\langle#1\right|}    % |#1>
\newcommand{\Xmark}{\text{\sffamily X}}        % cross out

\let\olditemize\itemize
\renewcommand{\itemize}{
  \olditemize
  \setlength{\itemsep}{1pt}
  \setlength{\parskip}{0pt}
  \setlength{\parsep}{0pt}
}


% Commands for temporary comments
\newcommand{\comment}[2]{\textcolor{red}{[\textbf{#1} #2]}}
\newcommand{\flip}[1]{{\color{red} [\textbf{Flip}: {#1}]}}
\newcommand{\email}[1]{\texttt{\href{mailto:#1}{#1}}}

\newenvironment{institutions}[1][2em]{\begin{list}{}{\setlength\leftmargin{#1}\setlength\rightmargin{#1}}\item[]}{\end{list}}


\usepackage{fancyhdr}		% to put preprint number



% Commands for listings package
%\usepackage{listings}      % \begin{lstlisting}, for code
%
% \lstset{basicstyle=\ttfamily\footnotesize,breaklines=true}
%    sets style to small true-type



%%%%%%%%%%%%%%%%%%%
%%%  HYPERREF  %%%%
%%%%%%%%%%%%%%%%%%%

%% This package has to be at the end; can lead to conflicts
\usepackage{microtype}
\usepackage[
	colorlinks=true,
	citecolor=black,
	linkcolor=black,
	urlcolor=green!50!black,
	hypertexnames=false]{hyperref}





\begin{document}


\begin{center}

    {\Large \textsc{Long HW 1}:
    \textbf{Kinematics and QED}}
    
\end{center}

\vskip .4cm

\noindent
\begin{tabular*}{\textwidth}{rl}
	\textsc{Course:}& Physics 165, \emph{Introduction to Particle Physics} (2022)
	\\
	\textsc{Instructor:}& Prof. Flip Tanedo (\email{flip.tanedo@ucr.edu})
	\\
	\textsc{Due by:}& {Thursday}, April 7
\end{tabular*}

\noindent
This is the main weekly homework set. Unless otherwise stated, give all responses in natural units where $c = \hbar = 1$ and energy is measured in electron volts (usually MeV or GeV). 

\subsection{Everything in natural units}

Write the following quantities in natural units with energy measured in GeV. You may write everything to one significant figure. The relevant constants (as well as everything we know about particle physics) is available free at the website of the Particle Data Group (\textbf{the PDG}), \url{http://pdg.lbl.gov}. Look at the pages for `Physical Constants' and `Astrophysical Constants.'

\begin{itemize}
	\item The mass of the sun, $M_\odot$.
	\item The present day Hubble expansion rate, $H_0$. 
	\item The classical electron radius, $r_e$. 
	\item The Schwarszchild radius of the sun, $2G_N M_\odot/c^2$.
\end{itemize}

\textsc{Comments}:
\begin{itemize} 
	\item If you're not familiar with what a Schwarzschild radius is, look it up. Even without knowing general relativity, you could have guessed the order of magnitude of the Schwarzschild radius simply from dimensional analysis. 
	\item What do you think the meaning of the classical electron radius is?
\end{itemize}

\subsection{Dynamics: QED}

\subsubsection{Compton Scattering}

\textbf{Compton scattering} is the process $e^- \gamma \to e^- \gamma$, an electron and a photon turning into an electron and a photon. You can assume that the incoming particles have different momenta than the outgoing particles so that some interaction must have happened. Draw \emph{all} of the \textbf{leading order} Feynman diagram(s) for Compton scattering in quantum electrodynamics.

\textsc{Comment}: recall that a valid diagram is (1) connected and (2) uses only the lines and vertices of QED. \emph{Leading order} means simplest: we want the diagrams with the fewest number of vertices.

\subsubsection{Inverse Compton Scattering}

A process called \emph{inverse} Compton scattering is important in astroparticle physics. What are the leading order Feynman diagrams for inverse Compton scattering? How is this process different from Compton scattering? 

\textsc{Comment}: Go ahead, use Google. If you find a source that explains something well, be sure to cite it.


\subsection{Special Relativity and Kinematics}

\subsubsection{The non-relativistic limit}

A particle is \textbf{non-relativistic} if its 3-momentum is much smaller than its energy, $|\vec{p}| \ll E$. In this limit, show that $E^2 = m^2 + \vec{p}^2$ reduces to the familiar $E=mc^2$ upon restoring factors of $c$. In ths non-relativistic limit, what is the \emph{leading-order correction} to $E=mc^2$? You'll want to Taylor expand---make sure you do this with respect to a small, \emph{dimensionless} parameter. (There's no such thing as a `small parameter' that has dimensions.)

\subsubsection{A relativistic electron}

In some frame, the electron has \emph{momentum} equivalent to its rest mass, $m_e$. Use the value of the rest mass to one significant figure. I shouldn't have to tell you where to look it up. Write out the components of the \textbf{momentum four-vector} $p^\mu$. 

\subsubsection{A symmetric particle collider}

Imagine a symmetric electron--positron collider. At the collision point, it collides a beam of electrons and positrons with one another so that these have four-momenta:
\begin{align}
	p^\mu_{e^-} &= (E,0,0,p)
	&
	\text{and}
	&&
	p^\mu_{e^+} &= (E,0,0,-p) \ .
\end{align}
What is the expression for $p$ as a function of $E$ and $m_e$? What is the \textbf{center of mass energy} of the collision in the lab frame?

Suppose that this collider was invented to produce a 91~GeV particle, $Z$, through the process $e^+ e^- \to Z$. What energy $E$ is required for each beam? What is the momentum of the $Z$ particle in the lab frame?

\textbf{Extra credit}: Suppose the $Z$ is unstable and decays. This means that you don't get to measure it directly. Without knowing anything else about how the $Z$ interacts, what is one \textbf{decay mode} that is guaranteed to exist? In other words, what types of particles should you make sure you can detect? 

\subsubsection{A fixed target experiment}


Imagine a very asymmetric kind of collider called a \textbf{fixed target experiment}: a high-energy beam of particles hits a stationary target. Assume that both the beam and the target are composed of protons and that the collision occurs head-on\footnote{This is a classical idea, but for now we can live with this kind of deceit. Relevant: \url{https://www.youtube.com/watch?v=AnaQXJmpwM4}}. Write the four-momenta of a beam particle and the target particle in the lab frame. 

Suppose you wanted to produce some completely made up particle---let's call it a \emph{Flippon}\footnote{Unrelated to this: \url{https://arxiv.org/abs/1602.01377}}---that has a mass of 14 GeV. To one significant figure, what proton beam energy $E$ is required to produce the Flippon through $pp \to \text{Flippon}$? (Assume that such a process is possible.) What is the momentum of the outgoing Flippon in the lab frame?

\textsc{Hint}: This problem is constrained by kinematics. There's an easy way and a hard way of doing this. The hard way is to perform a boost to the center of mass frame where it's easy. The easy way is to identify that the Einstein relation tells us that squares of momenta are \emph{invariant}, and so you don't need to do the boost. 

\textsc{Discussion}: Fixed target experiments are nice because you don't have to worry about engineering two beams to collide with one another. They also have a very useful feature that the new particle is produced \emph{boosted} relative to the lab frame. This can be very useful for untangling the decay products of the new particle from other  particle debris from the beam hitting the target.


%\subsubsection{Lorentz Transformations}
%
%What is the relativistic boost factor $\gamma$ of an electron that has been boosted to $E = 1$ GeV? What is the velocity of such an electron? Write out the boost matrix that transforms from the particle's rest frame to the frame where it has $E= 1$ GeV.
%
%\textsc{Hint}: If only there were some reference material in the back of some booklet that had a review of kinematics. 





\subsection{Feedback}

Approximately how long did it take you to complete the non-extra credit parts of this assignment?

\textsc{Comment}: Homework will become more challenging as the quarter progresses.



\section{Extra Credit}

If you do any of these problems, please write a short note giving your thoughts on the reading: did you like them? Were they too simple / difficult? I do not expect you to be able to complete all (or necessarily any) of the extra credit. You can access all journal articles through the UCR library.

% \subsection{Apply for an REU (or related)}

% Apply for a summer research experience for undergraduates (REU). Attach a copy of your application (at least the research proposal/statement of purpose---something \emph{you} wrote). If you are not eligible to apply for an REU, you may attach an application for graduate school, future employment, or other type of award that involves presenting your technical abilities. 

\subsection{Minkowski Diagrams}

The mathematical basis of relativity is geometry. This is most simply seen in what are called \textbf{Minkowski diagrams}. I'm pretty sure  A good introduction to these are in \url{https://arxiv.org/abs/1508.01968} by Boxiang Liu and Thushara Perera\footnote{A somewhat more polished reference is the book \emph{Very Special Relativity} by Sander Bais. There's also a Khan Academy video, \url{https://youtu.be/nEqexIckVCM}.}. Consider two reference frames with some non-zero relative velocity. Sketch the axes of the Minkowski diagram this system: that is, draw the $(x,t)$ axes and the $(x',t')$ axis where $(x',t')$ are related to $(x,t)$ by a Lorentz transformation.  Draw two spacetime events and their respective light cones. Comment on the idea of causality using these diagrams. Those who are mathematically inclined may enjoy \url{https://doi.org/10.1119/1.4997027}. 

\subsection{Impact of Special Relativity on Physics: Compton Scattering}

Look over David Jackson's article ``The Impact of Special Relativity on Theoretical Physics'' from the May 1987 issue of \emph{Physics Today}, \url{https://doi.org/10.1063/1.881108}. Focus on the section ``Waves and particles,'' where the author discusses \textbf{Compton scattering}.  Use special relativity to derive the author's expression for $\delta\lambda$, the shift in the photon wavelength.

\subsection{Relativistic mass}

There is an antiquated notion of \emph{relativistic mass} that people used to talk about. Lev Okun gives a nice overview in ``The Concept of Mass'' in the June 1989 issue of \emph{Physics Today}, \url{https://doi.org/10.1063/1.881171}. Read the article and explain why there is only \emph{one} useful notion of mass and that it is the \emph{rest mass}.


\end{document}